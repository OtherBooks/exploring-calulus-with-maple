\section{Orthogonal Curves}
\label{sec:orthogonal_curves}

\subsection*{Recommended Tutorials:}
\begin{itemize}[noitemsep]
	\item \nameref{chp:equation_solvers}, pg. \pageref{chp:equation_solvers}
	\item \nameref{chp:implicit_functions}, pg. \pageref{chp:implicit_functions}
\end{itemize}

\subsection*{Introduction:}

Orthogonal curves are curves that are perpendicular whenever they intersect. Perpendicular lines\index{lines!perpendicular lines} are an elementary example of this. We know that if $m_1$ and $m_2$ are the slopes of two perpendicular lines, then $m_1 m_2 = -1$. Similarly, two curves are orthogonal if their derivatives multiply to $-1$ whenever they intersect.

\marginnote[-6cm]{There are a few important things to remember about the \texttt{implicitplot()} command during this activity:\index{implicit functions!implicitplot}
	\begin{itemize}
	\item The \texttt{plots} package needs to be loaded using the \texttt{with()} command. 
	\item Some versions of Maple may not produce a smooth plot. In this case, include either \texttt{numpoints=30000} or \texttt{grid=[250,250]} as a parameter.
	\item The optional \texttt{scaling=constrained}\index{implicit functions!implicitplot!scaling} parameter can be included to enforce $1:1$ scaling. Alternatively, the optional scaling can be performed by clicking on the graph and then clicking on the $1:1$ button in the plot toolbar at the top of the page.
	\item If you are plotting multiple graphs on the same set of axes, it is a good idea to specify plot colours.\index{implicit functions!implicitplot!multiple plots}\index{implicit functions!implicitplot}
	\end{itemize}
}
\vspace{1cm}
\subsection*{Exercises:}

\begin{enumerate}
	\marginnote{Do not forget to include multiplication between $x$ and $y$.}
    \item Consider the curves $y^2 - x^2 = 3$ and $xy = 2$.
    \begin{enumerate}
    	\item Plot both of these curves on the same set of axes. Make sure both curves appear smooth.
    	\marginnote[-0.4cm]{In order to find points of intersection, Maple can solve a system of equations in one \texttt{solve()} \index{solving equations!solve} or \texttt{fsolve()}\index{solving equations!fsolve} command for both $x$ and $y$. If you choose to use the \texttt{solve()} command, you may need to include the optional parameter \texttt{explicit=true} to avoid the \textit{RootOf()} output.}
    	\item Find the $x$- and $y$-coordinates of the points of intersection.
    	\item Compute the derivatives of both curves.
    	\marginnote[0.3cm]{A similar example is detailed on page \pageref{subsec:orthocurves}.}
    	\item Are these curves orthogonal at the points of intersection? Confirm this using the fact that $m_1 m_2 = -1$ for perpendicular slopes.
    \end{enumerate}
    \index{solving equations!removing \texttt{RootOf()}}
    \vspace{0.8cm}
    \item Consider the two families of curves given by \[ y = cx^2 \] and \[ x^2 + 2y^2 = k, \] where $c$ and $k > 0$ are arbitrary constants. 
    \marginnote[0.8cm]{Be sure to include $y=$ when using the \texttt{implicitplot()} command to plot the curves for the first family.}
    \begin{enumerate}
    	\item Plot several curves of the first family using 
    	\[ c=-2,-1,0,1,2 \] 
    	\marginnote[0.2cm]{You may wish to make all curves of the same family the same colour.}
    	and several curves of the second family using
    	\[ k=1,4,9 \]
    	on the same axes.
    	\item Find the $x$ and $y$ coordinates of the four intersection points of the two families of curves. The coordinates of the points should be given in terms of $c$ and $k$.
    	\item Compute the derivative of each family of curves. 
    	\marginnote{It is a good idea to use the assignment operator $:=$ to assign each derivative in part (c) to a name. This will allow you to easily reference the derivatives when completing part (d).}
    	\item Show that the two families of curves are orthogonal at each of the four intersection points, regardless of $c$ and $k$.
    \end{enumerate}
\end{enumerate}